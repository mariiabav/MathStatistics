\newpage
\listoffigures

\newpage
\listoftables

\newpage
\section{Постановка задачи}
Сгенерировать двумерные выборки размерами 20, 60, 100 для нормального двумерного распределения $N(x, y, 0, 0, 1, 1, \rho)$. Коэффициент корреляции $\rho$ взять равным 0, 0.5, 0.9.
Каждая выборка генерируется 1000 раз и для неё вычисляются: среднее значение, среднее значение квадрата и дисперсия коэффициентов корреляции Пирсона, Спирмена и квадрантного коэффициента корреляции.
Повторить все вычисления для смеси нормальных распределений:
\begin{equation} \label{eq:f(x, y)}
f(x, y) = 0.9N(x, y, 0, 0, 1, 1, 0.9) + 0.1N(x, y, 0, 0, 10, 10, -0.9)
\end{equation}
Изобразить сгенерированные точки на плоскости и нарисовать эллипс
равновероятности.
\addcontentsline{toc}{section}{Постановка задачи}

\section{Теория}
\subsection{Двумерное нормальное распределение}
Двумерное нормальное распределение является частным случаем многомерного нормального распределения. Плотность вероятности двумерной случайной величины $X, Y$, распределённой нормально, выражается формулой:
\begin{equation} \label{eq:N(x, y...)} 
N(x, y, \overline{x}, \overline{y}, \sigma_x, \sigma_y, \rho) = 
\frac{1}{2 \pi \sigma_x \sigma_y \sqrt{1-\rho^2}} 
exp (-\frac{1}{2(1-\rho^2)} [\frac{(x-\overline{x})^2}{\sigma_x^2}
- 2\rho\frac{(x-\overline{x})(y-\overline{y})}{\sigma_x \sigma_y} + \frac{(y-\overline{y})^2}{\sigma_y^2}])
\end{equation}
Параметр $\rho$ называется коэффициентом корреляции.

\subsection{Корреляционный момент (ковариация) и коэффициент корреляции}
Ковариация — мера линейной зависимости двух случайных величин. Ковариация двух случайных величин $X$ и $Y$, определённых на одном и том же вероятностном пространстве, определяется формулой:
\begin{equation} \label{eq:K}
K = cov(X, Y) = M[(X-\overline{x})(Y-\overline{Y})] 
\end{equation}
Корреляция — взаимосвязь двух или более случайных величин. Математической мерой корреляции двух случайных величин $X$ и $Y$ служит коэффициент корреляции $\rho$, который определяется отношением:
\begin{equation} \label{eq:K}
\rho = \frac{K}{\sigma_x \sigma_y}
\end{equation}
Коэффициент корреляции изменяется в пределах от минус единицы до плюс единицы.

\subsection{Выборочные коэффициенты корреляции}
\subsubsection{Выборочный коэффициент корреляции Пирсона}
Естественной оценкой для $\rho = \frac{cov(X,Y)}{\sqrt{DXDY}}$ служит его статистический аналог в виде выборочного коэффициента корреляции, предложенного К.Пирсоном, —
\begin{equation} \label{eq:r}
r = \frac{1/n\sum(x_i-\overline{x})(y_i-\overline{y})} 
{1/n\sum (x_i-\overline{x})^2 1/n\sum (y_i-\overline{y})^2} 
= K/(s_Xs_Y),
\end{equation}
где $K, s^2_X, s^2_Y$ — выборочные ковариация и дисперсии с.в. $X$ и $Y$. \\ \\
Коэффициент корреляции Пирсона характеризует существование линейной зависимости между двумя величинами. Коэффициент также называют также теснотой линейной связи. \cite{theory}

\subsubsection{Выборочный квадрантный коэффициент корреляции}
Кроме выборочного коэффициента корреляции Пирсона, существуют и другие оценки степени взаимосвязи между случайными величинами. К ним относится выборочный квадрантный коэффициент корреляции 
\begin{equation} \label{eq:r}
r_Q = \frac{(n_1+n_3)-(n_2+n_4)} 
{n},
\end{equation}
где $n_1, n_2, n_3, n_4$ — количества точек с координатами $(x_i, y_i)$, попавшими соответственно в I, II, III и IV квадранты декартовой системы с осями $x^{'} = x -  med x, y^{'} = y -  med y$ и с центром в точке с координатами $(med x, med y)$. \cite{theory}


\subsubsection{Выборочный коэффициент ранговой корреляции Спирмена}
Если требуется оценить степень взаимодействия между качественными признаками изучаемого объекта, для оценки силы связи используются не численные значения, а соответствующие им ранги. Сам процесс упорядочения называется ранжированием.\\ \\
Если объект обладает не одним, а двумя качественными признаками — переменными $X, Y$, то для исследования их взаимосвязи используют выборочный коэффициент корреляции между двумя последовательностями рангов этих признаков. \\ \\
Обозначим ранги, соотвествующие значениям переменной $X$, через $u$, а ранги, соотвествующие значениям переменной $Y$ — через $v$. Выборочный коэффициент ранговой корреляции Спирмена определяется как выборочный коэффициент корреляции Пирсона между рангами $u, v$ переменных $X, Y$:
\begin{equation} \label{eq:r_S}
r_S = \frac{1/n\sum(u_i-\overline{u})(v_i-\overline{v})} 
{\sqrt{1/n\sum (u_i-\overline{u})^2 1/n\sum (v_i-\overline{v})^2}},
\end{equation}
где $\overline{u} = \overline{v} = \frac{1 + 2 + ... + n}{n} = \frac{n+1}{2}$ — среднее значение рангов. \cite{theory}

\subsection{Эллипсы рассеивания}
Поверхность распределения, изображающую данную функцию, имеет вид холма, вершина которого находится над точкой $(\overline{x}, \overline{y})$. В сечении поверхности распределения плоскостями, параллельными плоскости $xOy$ получаются эллипсы. \cite{ellipse_theory} Уравнение проекции такого эллипса на плоскость $xOy$:
\begin{equation} \label{eq:const}
\frac{(x-\overline{x})^2}{\sigma_x^2}
- 2\rho\frac{(x-\overline{x})(y-\overline{y})}{\sigma_x \sigma_y} + \frac{(y-\overline{y})^2}{\sigma_y^2} = const
\end{equation}
Осей симметрии эллипса составляют с осью  $Ox$ углы, определяемые уравнением
\begin{equation} \label{eq:const}
tg 2\alpha = \frac{2\rho\sigma_x\sigma_y}{\sigma_x^2 - \sigma_y^2} 
\end{equation} 
Во всех точках каждого из таких эллипсов плотность распределения $N(x, y, \overline{x}, \overline{y}, \sigma_x, \sigma_y, \rho)$ постоянна. Поэтому такие эллипсы называются называется эллипсами равной плотности, ограниченная ими область – эллипсами рассеивания, центры эллипсов – центрами рассеивания.


\section{Реализация}
Лабораторная работа была выполнена с помощью встроенных средств языка программирования Python в среде разработки IDLE. Исходный код лабораторной работы приведён по ссылке. В ходе работы использовались библиотеки Math, Matplotlib, Numpy и Seaborn. \\ \\
Помимо основных в ходе работы были использованы следующие инструменты:
\begin{itemize}
\item $multivariate.normal$ : генерация двумерных выборк размерами 20, 60, 100 для нормального двумерного распределения \cite{ellipse}
\item stats.pearsonr : вычисление коэффициента корреляции Пирсона
\item stats.spearmanr:  вычисление коэффициента корреляции Спирмена
\item $confidence.ellipse$ : построение эллипсов рассеивания
\end{itemize}

\section{Результаты}
\subsection{Выборочные коэффициенты корреляции}
