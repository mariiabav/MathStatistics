\section{Обсуждение}
Упорядочим полученные характеристики коэффициентов корреляции Пирсона, Спирмена и квадрантного коэффициента корреляции:
\begin{enumerate}
    \item Для двумерного нормального распределения и для смеси нормальных распределений среднее значение, среднее значение квадрата  упорядочены следующим образом: $ r_Q < r_S < r $
    \item Для двумерного нормального распределения и для смеси нормальных распределений дисперсии упорядочены следующим образом: $ r < r_S < r_Q $
\end{enumerate}

При построении эллипсов рассеивания с помощью функции confidence.ellipse радиусы эллипса можно контролировать с помощью входного параметра. Значение параметра по умолчанию позволяет эллипсу охватывать 99,7 процентов точек, что и видно на графиках.

\section{Приложение}
Исходный код текста программ можно найти по ссылке \\ https://github.com/mariiabav/MathStatistics.git 